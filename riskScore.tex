\chapter{Risk Rating System}
The Common Vulnerability Scoring System v3.0 (CVSS) will be used in the assessment to characterize and measure the vulnerability's potential impact on our CMS system \citep[p.~516]{cvss}. The CVSS has been adopted and used by many enterprises, i.e., National Vulnerability Database \citep{nist}. The rankings for vulnerabilities are assigned based on the score \citep{nvd_2022_vuln_calc}, which is as follows:

\begingroup
\centering
\setlength{\tabcolsep}{6.5pt} % Default value: 6pt
\renewcommand{\arraystretch}{1.8} % Default value: 1
\begin{longtable}{ |p{5cm}| p{10cm} |}
\caption{CVSS Severity Ratings}
    \label{table:spoofing}
\hline
\rowcolor{grey!15}
\textbf{Rating}  & \textbf{Description}\\
\hline
\cellcolor{red!95} Critical (9.0 - 10.0)  &  Test\\
\hline
\cellcolor{red!70} High (7.0-8.9)  &  Test\\
\hline
\hline
\cellcolor{yellow!95} Medium (4.0-6.9)  &  Test\\
\hline
\cellcolor{green!95} low (0.1-3.9)  &  Test\\
\hline
\cellcolor{grey!55} None (0.0)  &  Test\\
\hline
\end{longtable}
\endgroup